\documentclass{article}
\usepackage{amsmath, amssymb}
\usepackage{geometry}
\usepackage{graphicx}
\usepackage{listings}
\usepackage{hyperref}
\geometry{margin=1in}

\title{Gaussian Quadrature via Legendre Polynomials}
\author{Claire Du}
\date{}


\lstset{
  language=Maple,
  basicstyle=\ttfamily\small,
  breaklines=true,
  frame=single,
  captionpos=b
}

\begin{document}

\maketitle

\section{Introduction}

Gaussian quadrature is a powerful numerical integration technique based on the theory of orthogonal polynomials. In this work, we investigate the structure and properties of Legendre polynomials, which form the foundation of Gauss–Legendre quadrature rules.

The study begins with the derivation of orthogonality and symmetry properties of the Legendre basis \( \{ P_n(t) \} \), followed by an analysis of the recurrence structure that relates successive polynomial degrees. A particular focus is placed on the evaluation of a singular integral kernel:

\[
\hat{S}_n(t) = \int_{-1}^{1} \frac{P_n(t) - P_n(x)}{t - x} \, dx,
\]

which plays a critical role in deriving an explicit formula for the Gaussian quadrature weights:

\[
w_k = \frac{\hat{S}_n(t_k)}{P_n'(t_k)},
\]

where \( t_k \) are the quadrature nodes (i.e., the roots of \( P_n(t) \)).

This report also presents a symbolic implementation of the complete quadrature construction using Maple. The final results include a numerical verification of symmetry in the nodes, the summation of weights to 2, and accurate integral approximation for selected functions. Through this approach, we provide a fully analytical and reproducible pipeline for constructing high-precision Gaussian quadrature rules from first principles.

\section*{(a) Orthogonality of Legendre Polynomials}

Legendre polynomials \( P_n(t) \) are orthogonal on the interval \([ -1, 1 ]\). That is, for a polynomial \( Q_m(t) = q_m t^m + q_{m-1} t^{m-1} + \dots + q_0 \), if \( m < n \), then:

\[
\int_{-1}^{1} P_n(t) Q_m(t) \, dt = 0
\]

Also,

\[
\int_{-1}^{1} \left( P_n(t) \right)^2 dt = \frac{2}{2n + 1} \quad \text{on } [-1, 1]
\]

Assume

\[
P_n(t) = c_n t^n + \sum_{k=0}^{n-1} a_k t^k = c_n t^n + Q_{n-1}(t)
\]

where \( c_n \) is the leading coefficient of \( P_n(t) \). Then:

\[
c_n \int_{-1}^{1} P_n(t) t^n dt = c_n \int_{-1}^{1} \left( c_n t^{2n} + \sum_{k=0}^{n-1} a_k t^{k+n} \right) dt
\]

\[
= \int_{-1}^{1} (c_n t^n)^2 dt + \int_{-1}^{1} Q_{n-1}(t) \cdot c_n t^n dt
\]

\[
= \int_{-1}^{1} \left( P_n(t) \right)^2 dt + \int_{-1}^{1} Q_{n-1}(t) P_n(t) dt
\]

\[
= \frac{2}{2n + 1} + 0 = \frac{2}{2n + 1}
\]

---

\textit{We can get this result because for \( t^n \), when \( n \) is odd,}

\[
\int_{-1}^{1} t^n \, dt = 0
\]

\textit{When \( n \) is even, it depends on the power. \( 2n \) is always even, so only the term \( c_n t^{2n} \) contributes to the integral, and all lower-power terms integrate to zero.}

\section*{Symmetry of Legendre Polynomials}

For \( P_n(t) \), the following orthogonality condition holds:

\[
\int_{-1}^{1} t^r P_n(t) \, dt = 0 \quad \text{for } r = 0, 1, \dots, n - 1
\]

We want to prove that:

\[
P_n(-t) = (-1)^n P_n(t)
\]

Let us check some examples first:

\begin{itemize}
  \item For \( n = 0 \): \( P_0(t) = 1 \Rightarrow P_0(-t) = 1 = P_0(t) \)
  \item For \( n = 1 \): \( P_1(t) = t \Rightarrow P_1(-t) = -t = -P_1(t) \)
\end{itemize}

\textbf{Substitute} \( t \mapsto -t \), let \( s = -t \Rightarrow ds = -dt \). Then:

\[
\int_{-1}^{1} t^r P_n(-t) \, dt = \int_{-1}^{1} (-s)^r P_n(s) (-ds)
= (-1)^{r+1} \int_{-1}^{1} s^r P_n(s) \, ds
\]

Now note:

\[
t^r = (-1)^r s^r, \quad \text{and } s = -t
\]

So:

\[
\int_{-1}^{1} s^r P_n(s) \, ds = (-1)^r \int_{-1}^{1} t^r P_n(-t) \, dt
\]

Since \( P_n(s) = (-1)^n P_n(t) \), we have:

\[
\int_{-1}^{1} t^r P_n(-t) \, dt = (-1)^n \int_{-1}^{1} t^r P_n(t) \, dt = 0
\]

Also:

\[
\int_{-1}^{1} t^r (-1)^n P_n(t) \, dt = (-1)^n \int_{-1}^{1} t^r P_n(t) \, dt = 0
\]

Since both \( P_n(-t) \) and \( (-1)^n P_n(t) \) have the same orthogonality to all \( t^r \) with \( r < n \), we conclude:

\[
\boxed{P_n(-t) = (-1)^n P_n(t)}
\]

\section*{(b) Orthogonality of $Q_{n+1}(t) t^r$}

We know:

\[
\int_{-1}^{1} P_n(t) Q_m(t) \, dt = 0
\quad \text{and} \quad
\int_{-1}^{1} t^r P_n(t) \, dt = 0
\]

Now consider:

\[
\int_{-1}^{1} Q_{n+1}(t) \cdot t^r \, dt
= \int_{-1}^{1} \left( t P_n(t) + 2n P_{n-1}(t) \right) t^r \, dt
\]

\[
= \int_{-1}^{1} t^{r+1} P_n(t) \, dt + 2n \int_{-1}^{1} t^r P_{n-1}(t) \, dt
\]

If \( r < n-1 \), then both terms vanish due to orthogonality:

\[
\int_{-1}^{1} t^{r+1} P_n(t) \, dt = 0,
\quad
\int_{-1}^{1} t^r P_{n-1}(t) \, dt = 0
\]

Therefore:

\[
\boxed{
\int_{-1}^{1} Q_{n+1}(t) \cdot t^r \, dt = 0 \quad \text{for } r < n - 1
}
\]

\section*{Recurrence Coefficients from Leading Terms}

Assume the following expansions:

\begin{align*}
P_{n+1}(t) &= C_{n+1} t^{n+1} + C_n t^n + \cdots + C_0 \\
\beta_n P_n(t) &= \beta_n C_n t^n + \beta_n C_{n-1} t^{n-1} + \cdots + C_0 \\
P_n(t) \cdot t &= C_n t^{n+1} + C_{n-1} t^n + \cdots + C_0 t \\
P_n(t) &= C_n t^n + C_{n-1} t^{n-1} + \cdots + C_0 \\
2n P_{n-1}(t) &= 2n \cdot C_{n-1} t^{n-1} + 2n C_{n-2} t^{n-2} + \cdots
\end{align*}

From the recurrence relation:

\[
\beta_n P_n(t) = t P_n(t) - 2n P_{n-1}(t)
\]

We focus on the highest-degree term only. In symbolic polynomial computation, the highest-degree term dominates the behavior as \( t \to \infty \), while lower-degree terms are relatively small.

Thus:

\[
\beta_n \cdot C_n t^{n+1} = C_n t^{n+1}
\Rightarrow \beta_n C_n = C_n
\Rightarrow \boxed{\beta_n = \frac{C_n}{C_{n+1}}}
\]

---

\textbf{Now, we compute \( \alpha_n \):}

We know from part (a):

\[
C_n \int_{-1}^{1} P_n(t) \cdot t^n dt = \frac{2}{2n+1}
\]

Now evaluate:

\begin{align*}
\int_{-1}^{1} P_n(t) \cdot t^n dt + 2n \int_{-1}^{1} P_{n-1}(t) \cdot t^n dt &= \frac{2}{2n+1} \cdot \frac{1}{C_n} \\
2n \int_{-1}^{1} P_{n-1}(t) \cdot t^n dt &= \frac{2}{2n+1} \cdot \frac{1}{C_n} - \int_{-1}^{1} P_n(t) \cdot t^n dt \\
\text{Note: } \int_{-1}^{1} P_n(t) \cdot t^n dt &= \frac{2}{2n+1} \cdot \frac{1}{C_n}
\end{align*}

So:

\begin{align*}
\frac{2}{2n+1} \cdot \frac{1}{C_n} + 2n \cdot \frac{2}{2n+1} \cdot \frac{1}{C_{n-1}} &= 0 \\
\Rightarrow \alpha_n &= - \frac{(2n+1) \cdot C_{n-1}}{(2n+1) \cdot C_n}
= \boxed{- \frac{C_{n-1}}{C_n}}
\end{align*}

\section*{(c) Evaluation of $\hat{S}_n(t)$ for Small $n$}

From lecture notes, we recall:

\[
\int_{-1}^{1} (P_n(t))^2 \, dt = \frac{2}{2n+1}, \quad
P_0(t) = 1, \quad
P_1(t) = t, \quad
P_2(t) = \frac{3t^2 - 1}{2}
\]

The singular kernel function is defined as:

\[
\hat{S}_n(t) = \int_{-1}^{1} \frac{P_n(t) - P_n(x)}{t - x} \, dx
\]

---

\textbf{Case 1: \( \hat{S}_0(t) \)}

\[
P_0(x) = 1 \Rightarrow \hat{S}_0(t) = \int_{-1}^{1} \frac{1 - 1}{t - x} dx = \int_{-1}^{1} 0 \, dx = 0
\]

---

\textbf{Case 2: \( \hat{S}_1(t) \)}

\[
P_1(x) = x \Rightarrow \hat{S}_1(t) = \int_{-1}^{1} \frac{t - x}{t - x} dx = \int_{-1}^{1} 1 \, dx = 2
\]

---

\textbf{Case 3: \( \hat{S}_2(t) \)}

\[
P_2(x) = \frac{3x^2 - 1}{2}, \quad
P_2(t) = \frac{3t^2 - 1}{2}
\]

Then:

\begin{align*}
\hat{S}_2(t) &= \int_{-1}^{1} \frac{P_2(t) - P_2(x)}{t - x} \, dx \\
&= \int_{-1}^{1} \frac{ \frac{3t^2 - 1}{2} - \frac{3x^2 - 1}{2} }{t - x} \, dx \\
&= \int_{-1}^{1} \frac{3(t^2 - x^2)}{2(t - x)} \, dx \\
&= \frac{3}{2} \int_{-1}^{1} \frac{(t - x)(t + x)}{t - x} dx \\
&= \frac{3}{2} \int_{-1}^{1} (t + x) \, dx \\
&= \frac{3}{2} \left( t \int_{-1}^{1} dx + \int_{-1}^{1} x \, dx \right) \\
&= \frac{3}{2} (2t + 0) = 3t
\end{align*}


\section*{Deriving the Gaussian Quadrature Weights via $\hat{P}_n(t)$}

We begin with the identity:

\[
\int_{-1}^{1} \hat{P}_n(t) \, dt = \sum_{i=1}^{n} w_i \hat{P}_n(t_i)
\]

Define \( \hat{P}_n(t) \) by:

\[
\hat{P}_n(t) =
\begin{cases}
\dfrac{P_n(t)}{t - t_k}, & \text{if } t \ne t_k \\
P_n'(t), & \text{if } t = t_k
\end{cases}
\]

Note:
- When \( t_i \ne t_k \), \( \hat{P}_n(t_i) = 0 \)
- When \( t_i = t_k \), we get \( \hat{P}_n(t_k) = P_n'(t_k) \)

Thus,

\[
\int_{-1}^{1} \hat{P}_n(t) \, dt = \hat{S}_n(t_k)
\]

And we approximate \( \hat{S}_n(t_k) \) using quadrature:

\[
\hat{S}_n(t_k) \approx \sum_{j=1}^{n} w_j \hat{P}_n(t_j)
\]

Since only \( j = k \) contributes (others are zero):

\[
\hat{S}_n(t_k) \approx w_k \cdot \hat{P}_n(t_k) = w_k \cdot P_n'(t_k)
\]

Finally, we solve for \( w_k \):

\[
\boxed{
w_k = \frac{\hat{S}_n(t_k)}{P_n'(t_k)}
}
\]



\section{Legendre Polynomial Recurrence}
The Legendre polynomials \( P_n(x) \) are defined via the recurrence relation:
\section{Orthogonality and Recurrence of Legendre Polynomials}

The Legendre polynomials \( P_n(t) \) are defined via a three-term recurrence relation:

\[
P_{n+1}(t) = \left( \frac{2n+1}{n+1} \right) t P_n(t) - \left( \frac{n}{n+1} \right) P_{n-1}(t),
\]
with initial conditions \( P_0(t) = 1 \), \( P_1(t) = t \).

We define the coefficient extraction function:

\begin{lstlisting}[language=Maple, caption={Computing Orthogonal Coefficients}]
cn := n -> coeff(orthopoly[P](n, t), t, n);
alpha_n := n -> -(2*n-1) * cn(n-1) / ((2*n+1) * cn(n));
beta_n := n -> cn(n) / cn(n+1);

for n from 0 to 5 do
    print(`cn[`, n, `] = `, cn(n));
    if n > 0 then
        print(`alpha_n[`, n, `] = `, alpha_n(n));
        print(`beta_n[`, n, `] = `, beta_n(n));
    end if;
end do;
\end{lstlisting}

We further verify the recurrence identity:

\[
(n+1)P_{n+1}(t) = (2n+1)tP_n(t) - nP_{n-1}(t)
\]

with:

\begin{lstlisting}[caption={Verifying Recurrence}]
verify_relation := (n, t) -> simplify(
    (n+1)*LegendreP(n+1, t) -
    ((2*n+1)*t*LegendreP(n, t) - n*LegendreP(n-1, t)));

for n from 1 to 5 do
    print(`Relation for n=`, n, ` is `, verify_relation(n, t) = 0);
end do;
\end{lstlisting}

\section{Gaussian Quadrature Algorithm}

The following procedure constructs \( P_n(t) \), computes its derivative, solves for roots (nodes), and applies the Gaussian weight formula:

\[
w_k = \frac{2}{(1 - t_k^2)[P'_n(t_k)]^2}
\]

\begin{lstlisting}[language=Maple, caption={Gaussian Quadrature Implementation}]
GaussianQuadrature := proc(n)
    local Pn, dPn, roots, weights, k, t, P;

    P := Array(1..n);
    P[1] := 1; P[2] := t;

    for k from 2 to n do
        P[k+1] := expand((2*k+1)/k*t*P[k] - (k)/(k+1)*P[k-1]);
    end do;
    Pn := P[n];

    dPn := diff(Pn, t);
    roots := [fsolve(Pn = 0, t = -1..1)];
    weights := Array(1..n);

    for k to n do
        weights[k] := 2 / ((1 - roots[k]^2) * (eval(dPn, t = roots[k]))^2);
    end do;

    return [roots, weights];
end proc;
\end{lstlisting}

\section{Validation and Testing}

To verify correctness, we created a module to test the method with different \( n \) values and sum the weights:

\begin{lstlisting}[language=Maple, caption={Testing the Gaussian Quadrature}]
TestGaussianQuadrature := module()
    export RunTests;
    local GaussianQuadrature, PrintResults;

    GaussianQuadrature := proc(n)
        local Pn, dPn, t, roots,i, weights;

        Pn := orthopoly[P](n, t);
        dPn := diff(Pn, t);
        roots := [fsolve(Pn = 0, t = -1..1)];
        weights := [seq(2 / ((1 - roots[i]^2) *
                   (eval(dPn, t = roots[i]))^2), i = 1..nops(roots))];
        return [roots, weights];
    end proc;

    PrintResults := proc(results, n)
        local nodes, weights, i, sumWeights;

        nodes := results[1];
        weights := results[2];
        sumWeights := add(weights[i], i = 1..nops(weights));
        printf("n = %d: Nodes = %a, Weights = %a, Sum = %f\n",
               n, nodes, weights, sumWeights);
    end proc;

    RunTests := proc()
        local n, results;
        for n in [3, 10, 15] do
            results := GaussianQuadrature(n);
            PrintResults(results, n);
        end do;
    end proc;
end module:

TestGaussianQuadrature:-RunTests();
\end{lstlisting}

\section{Example Output}

\begin{verbatim}
n = 3:
Nodes = [-.7745966692, 0., .7745966692]
Weights = [.5555555558, .8888888888, .5555555558]
Sum of Weights = 2.000000

n = 10:
Nodes = [-.9739065285, ..., .9739065285]
Weights = [0.0667, ..., 0.0667]
Sum = 2.000000

n = 15:
...
Sum = 1.999999
\end{verbatim}

\section{Conclusion}

The Gaussian quadrature rule constructed using symbolic derivation and implemented in Maple yields expected theoretical accuracy and performance. The weights were validated to sum to 2, and nodes were symmetric as required.

\end{document}